% Options for packages loaded elsewhere
\PassOptionsToPackage{unicode}{hyperref}
\PassOptionsToPackage{hyphens}{url}
%
\documentclass[
  11pt,
  ngerman,
  a4paper,
  oneside]{article}
\usepackage{lmodern}
\usepackage{amssymb,amsmath}
\usepackage{ifxetex,ifluatex}
\ifnum 0\ifxetex 1\fi\ifluatex 1\fi=0 % if pdftex
  \usepackage[T1]{fontenc}
  \usepackage[utf8]{inputenc}
  \usepackage{textcomp} % provide euro and other symbols
\else % if luatex or xetex
  \usepackage{unicode-math}
  \defaultfontfeatures{Scale=MatchLowercase}
  \defaultfontfeatures[\rmfamily]{Ligatures=TeX,Scale=1}
\fi
% Use upquote if available, for straight quotes in verbatim environments
\IfFileExists{upquote.sty}{\usepackage{upquote}}{}
\IfFileExists{microtype.sty}{% use microtype if available
  \usepackage[]{microtype}
  \UseMicrotypeSet[protrusion]{basicmath} % disable protrusion for tt fonts
}{}
\makeatletter
\@ifundefined{KOMAClassName}{% if non-KOMA class
  \IfFileExists{parskip.sty}{%
    \usepackage{parskip}
  }{% else
    \setlength{\parindent}{0pt}
    \setlength{\parskip}{6pt plus 2pt minus 1pt}}
}{% if KOMA class
  \KOMAoptions{parskip=half}}
\makeatother
\usepackage{xcolor}
\IfFileExists{xurl.sty}{\usepackage{xurl}}{} % add URL line breaks if available
\IfFileExists{bookmark.sty}{\usepackage{bookmark}}{\usepackage{hyperref}}
\hypersetup{
  pdftitle={Vignette for package yart},
  pdfauthor={Sebastian Sauer},
  pdflang={de-De},
  hidelinks,
  pdfcreator={LaTeX via pandoc}}
\urlstyle{same} % disable monospaced font for URLs
\usepackage[margin=1in]{geometry}
\usepackage{graphicx,grffile}
\makeatletter
\def\maxwidth{\ifdim\Gin@nat@width>\linewidth\linewidth\else\Gin@nat@width\fi}
\def\maxheight{\ifdim\Gin@nat@height>\textheight\textheight\else\Gin@nat@height\fi}
\makeatother
% Scale images if necessary, so that they will not overflow the page
% margins by default, and it is still possible to overwrite the defaults
% using explicit options in \includegraphics[width, height, ...]{}
\setkeys{Gin}{width=\maxwidth,height=\maxheight,keepaspectratio}
% Set default figure placement to htbp
\makeatletter
\def\fps@figure{htbp}
\makeatother
\setlength{\emergencystretch}{3em} % prevent overfull lines
\providecommand{\tightlist}{%
  \setlength{\itemsep}{0pt}\setlength{\parskip}{0pt}}
\setcounter{secnumdepth}{5}
\ifxetex
  % Load polyglossia as late as possible: uses bidi with RTL langages (e.g. Hebrew, Arabic)
  \usepackage{polyglossia}
  \setmainlanguage[]{german}
\else
  \usepackage[shorthands=off,main=ngerman]{babel}
\fi
\usepackage[]{biblatex}

\title{Vignette for package yart}
\usepackage{etoolbox}
\makeatletter
\providecommand{\subtitle}[1]{% add subtitle to \maketitle
  \apptocmd{\@title}{\par {\large #1 \par}}{}{}
}
\makeatother
\subtitle{NOT via an r package, but as a pandoc-template}
\author{Sebastian Sauer}
\date{12. 04. 2021}

\begin{document}
\maketitle
\begin{abstract}
Yart provides an RMarkdown template for rendering TeX based PDFs. It
provides a format suitable for academic settings. The typical RMarkdown
variables may be used. In additiion, some variabels useful for academic
reports have been added such as name of referee, due date, course title,
field of study, addres of author, and logo, and a few more maybe. In
addition, paper format (eg., paper size, margins) may be adjusted; the
babel language set of Latex is supported. Those variables are defined in
the yaml header of the yart document. Adjust those variables to your
need. Note that citations, figure/ table referencing is possible due to
the underlying pandoc magic. This template is not much more than setting
some of the variables provided by rmarkdown (pandoc, knitr, latex, and
more), credit is due to the original authors. Please reade the rmarkdown
documentation for detailled information on how to use rmarkdown and how
to change settings.
\end{abstract}

\hypertarget{my-section-header-1}{%
\section{My Section Header 1}\label{my-section-header-1}}

Please see the documentation of
\href{http://rmarkdown.rstudio.com/}{RMarkdown} for more details on how
to write RMarkdown documents.

Download a testlogo from here:
\url{https://raw.githubusercontent.com/sebastiansauer/yart/master/docs/logo.png}
and uncomment the respective line in the header.

For finetuning of design options, please check the tex template. There
you will find some variables such as \texttt{\$classoption\$}. Those
variables may be addressed in the yaml header of the yart file.

\hypertarget{my-section-header-2}{%
\subsection{My Section Header 2}\label{my-section-header-2}}

``Lorem ipsum'' dolor sit amet, consectetur adipiscing elit. Proin
mollis dolor vitae tristique eleifend. Quisque non ipsum sit amet velit
malesuada consectetur. Praesent vel facilisis leo. Sed facilisis varius
orci, ut aliquam lorem malesuada in. Morbi nec purus at nisi fringilla
varius non ut dui. Pellentesque bibendum sapien velit. Nulla purus
justo, congue eget enim a, elementum sollicitudin eros. Cras porta augue
ligula, vel adipiscing odio ullamcorper eu. In tincidunt nisi sit amet
tincidunt tincidunt. Maecenas elementum neque eget dolor
\href{http://example.com}{egestas fringilla}:

\printbibliography

\end{document}
